\documentclass[10pt,letterpaper]{article}
\usepackage[top=0.85in,left=2.75in,footskip=0.75in,marginparwidth=2in]{geometry}

% use Unicode characters - try changing the option if you run into troubles with special characters (e.g. umlauts)
\usepackage[utf8]{inputenc}

% clean citations
\usepackage{cite}

% hyperref makes references clicky. use \url{www.example.com} or \href{www.example.com}{description} to add a clicky url
\usepackage{nameref,hyperref}

% line numbers
\usepackage[right]{lineno}

% improves typesetting in LaTeX
\usepackage{microtype}
\DisableLigatures[f]{encoding = *, family = * }

% text layout - change as needed
\raggedright
\setlength{\parindent}{0.5cm}
\textwidth 5.25in 
\textheight 8.75in

% Remove % for double line spacing
%\usepackage{setspace} 
%\doublespacing

% use adjustwidth environment to exceed text width (see examples in text)
\usepackage{changepage}

% adjust caption style
\usepackage[aboveskip=1pt,labelfont=bf,labelsep=period,singlelinecheck=off]{caption}

% remove brackets from references
\makeatletter
\renewcommand{\@biblabel}[1]{\quad#1.}
\makeatother

% headrule, footrule and page numbers
\usepackage{lastpage,fancyhdr,graphicx}
\usepackage{epstopdf}
\pagestyle{myheadings}
\pagestyle{fancy}
\fancyhf{}
\rfoot{\thepage/\pageref{LastPage}}
\renewcommand{\footrule}{\hrule height 2pt \vspace{2mm}}
\fancyheadoffset[L]{2.25in}
\fancyfootoffset[L]{2.25in}

% use \textcolor{color}{text} for colored text (e.g. highlight to-do areas)
\usepackage{color}

% define custom colors (this one is for figure captions)
\definecolor{Gray}{gray}{.25}

% this is required to include graphics
\usepackage{graphicx}

% use if you want to put caption to the side of the figure - see example in text
\usepackage{sidecap}

% use for have text wrap around figures
\usepackage{wrapfig}
\usepackage[pscoord]{eso-pic}
\usepackage[fulladjust]{marginnote}
\reversemarginpar

% Adding multirow.
\usepackage{multirow}

% Other required things:
\usepackage{color}
\usepackage{subcaption}
\captionsetup[subfigure]{justification=centering}
\newcommand{\beachmat}{\textit{beachmat}}
\newcommand{\code}[1]{\texttt{#1}}

\newcommand{\suppfigsimpleaccess}{1}
\newcommand{\suppfigsparseschem}{2}
\newcommand{\suppfigsparsecol}{3}
\newcommand{\suppfighdflayout}{4-5}
\newcommand{\suppfigtenx}{6}

\newcommand{\suppseclayoutoptim}{1}
\newcommand{\suppseclayouttest}{2}

% document begins here
\begin{document}
\vspace*{0.35in}

% title goes here:
\begin{flushleft}
{\Large
    \textbf\newline{On the correct detection of empty droplets in droplet-based single-cell RNA sequencing protocols}
}
\newline

% authors go here:
%\\
Aaron T. L. Lun\textsuperscript{1,*},
Samantha Riesenfeld\textsuperscript{2,*},
Tallulah Andrews\textsuperscript{3,*},
The Phuong Dao\textsuperscript{4,*},
Tomas Gomes\textsuperscript{3,*}
and others
\\
\bigskip
\bf{1} Cancer Research UK Cambridge Institute, University of Cambridge, Li Ka Shing Centre, Robinson Way, Cambridge CB2 0RE, United Kingdom \\
\bf{2} Something... Broad? \\
\bf{3} Wellcome Trust Sanger Institute, Wellcome Genome Campus, Hinxton, Cambridge CB10 1SA, United Kingdom \\
\bf{4} Something... Columbia?
\\
\bigskip
* These authors contributed equally to this work.

\end{flushleft}

\section*{Introduction}
Recent advances in droplet-based protocols have revolutionized the field of single-cell transcriptomics by allowing tens of thousands of cells to be profiled in a single assay \cite{macosko2015highly,klein2015droplet,zheng2017massively}.
In these technologies, individual cells are captured into aqueous droplets in a water-in-oil emulsion.
Each droplet also contains a co-captured bead with primers for reverse transcription, where all primers on a single bead contain a cell barcode that is (effectively) unique to that bead.
The droplets serve as isolated reaction chambers in which cell lysis and reverse transcription are performed to obtain barcoded cDNA.
This is followed by breaking of the emulsion, amplification of the cDNA and construction of a sequencing library.
After sequencing, debarcoding is performed based on the cell barcode observed in each read sequence.
This yields an expression profile for each cell, typically in the form of unique molecular identifier (UMI) counts \cite{islam2014quantitative} for all annotated genes. 
The use of droplets increases throughput by at least an order of magnitude compared to protocols based on plates \cite{picelli2013smartseq2} or conventional microfluidics \cite{pollen2014low}, which is appealing for large-scale projects such as the Human Cell Atlas \cite{regev2017human}.

That said, the complexity of the sequencing data from droplet-based technologies poses a number of interesting challenges for low-level data processing.
One such challenge is the identification and removal of cell barcodes corresponding to empty droplets.
An empty droplet does not contain a cell but will still contain ``ambient'' RNA \cite{macosko2015highly}, i.e., cell-free transcripts in the solution in which the cells are suspended.
Ambient RNA may be actively secreted by cells or released upon cell lysis, the latter of which is particularly likely given the stresses of dissociation.
The presence of ambient RNA means that many empty droplets will contain material for reverse transcription and library preparation, resulting in non-zero total UMI counts for the corresponding barcodes.
However, the resulting expression profiles do not originate from any single cell and need to be removed prior to further analysis to avoid misleading conclusions.

Existing methods for removing empty droplets assume that droplets containing genuine cells should have more RNA, resulting in larger total UMI counts for the corresponding barcodes.
Zheng \textit{et al.} \cite{zheng2017massively} remove all barcodes with total counts below 10\% of the 99\textsuperscript{th} percentile of the $Y$ largest total counts (where $Y$ is defined as the expected number of cells to be captured on the Chromium device).
Macosko \textit{et al.} \cite{macosko2015highly} set the threshold at the inflection point in the cumulative fraction of reads with respect to increasing total count.
While simple, the use of a one-dimensional filter on the total UMI count is suboptimal as it may discard small cells with low RNA content.
Droplets containing small cells may not be easily distinguishable from large empty droplets in terms of the total number of transcripts.
This problem is exacerbated by variable capture and amplification efficiency across droplets, which further mixes the distributions of total counts between empty and non-empty droplets.
A simple threshold on the total count forces the researcher into a difficult choice between the loss of small cells or an increase to the number of artifactual ``cells'' composed of ambient RNA.

In this report, we propose a new method for detecting empty droplets in droplet-based single-cell RNA sequencing (scRNA-seq) data.
We construct a profile of the ambient pool of RNA, and test each barcode for deviations from this profile using a Poisson-based model for the count distribution.
Barcodes with significant deviations are considered to be genuine cells, thus allowing recovery of cells with low total RNA content and small total UMI counts.
We combine our approach with an inflection point filter to ensure that barcodes with large total counts are always retained.
Using a variety of simulations, we demonstrate that our method outperforms any simple threshold on the total UMI count.
We also apply our method to several real data sets where we are able to recover more cells from both existing and new cell types.

\section*{Description of the method}
These protocols also use unique molecular identifiers (UMIs) 

\section*{Results}

\section*{Discussion}

\section*{Methods}

\bibliography{ref.bib}
\bibliographystyle{unsrt}

\end{document}
